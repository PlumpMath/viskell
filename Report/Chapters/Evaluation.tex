\chapter{Evaluation}
\label{chap:Evaluation}

\section{Product}

The final product is close to what we wanted to build. However, due to lack of time some features where changed to be easier and some features were dropped from this release.

We are happy with the way the user interface shows types and the way the type checker works. The simple user interface also turned out the way we hoped to.

Our target audience includes people who are familiar with Haskell. At this moment Viskell cannot completely satisfy these people. Missing features include a compact representation for larger programs, the ability to save and load programs and functionality for directly inputting Haskell code. Another thing is that there is no support for I/O or libraries besides Prelude. It is possible to add this functionality at a later point.

We would have liked to make the program more useful for a large touch table where multiple users are interacting with the software at the same time. The circle menu would have been great for this purpose. Since our idea and prototype is described in this document we are confident that this can be implemented once this functionality is required.

\section{Process}

On the technical tools side, none of the members on the team had extensive experience with git \emph{and} GitHub \emph{and} Maven \emph{and} Travis, and some had never used either. It took quite a while for some to `get' the proposed workflow, and there was a lot of friction, many hours lost, and some frustration with those who struggled. Presentations on the subject helped, but some friction remained.

On a few occasions, we had difficulties planning sprints. Sprint goals and requirements either turned out to be vague at the end of the sprint, making it unclear or disputed whether or not we had completed them, or the goals were missed altogether. There also were times where we were not able to decide on how to implement certain features, which took up a lot of time.

The familiarity with the tools grew during the project, and all members became more comfortable with the tools. We also saw improvement in the meetings, sprints got defined clearer and a consensus was reached sooner. This improvement had a significant impact on the (decreased) duration of meetings.

Overall we think that sometimes too much time was invested in `overhead' and not directly into development of the project. Our opinion is that this is partially caused by the lack of management training beforehand. We would like to see that the replacement for this course in the TOM (Twents Onderwijs Model) education addresses this.


