\chapter{Implementation}

\section{Front end}

\subsection{Graphical design}

\subsection{Behaviour and interaction design}

\section{Back end}

\subsection{GHC integration}

The front end enables the user to construct Haskell expressions in a convenient manner.
The visual representation is converted into a tree of expression (Expr) objects and passed to the back end. \index{Expr}
The back end, finally, integrates with the Glasgow Haskell Compiler (\gls{GHC}) to do the actual computation.

The GHC integration is very simplistic.
The back end launches an instance of GHCi, the interactive read-eval-print-loop. \index{REPL}
This \gls{REPL} is then drip-fed the expressions over its standard input stream, converted from the intermediate representation into actual Haskell program code.
If the expression compiles and executes without problems, the result is returned to the front end.

\subsection{Type checker}

From our own experience as beginning Haskell programmers, we often encountered type errors, and therefore GHC's type error messages.
Very early in the project, we decided that the back end had to do at least some type-related work itself, to provide better errors to the user, and also to be able to show type hints, without having to parse GHC's error messages.
Not long thereafter, it was decided that such a home-grown type checker would not need to support the entirety of Haskell's type system.

After some iterations, we have arrived at an approach that could probably be described as a subset of Hindley-Milner type inference. \index{Hindley-Milner} \index{type inference}

Our (Java) implementation of Hindley-Milner type inference is based on an implementation in Scala by Andrew Forrest\cite{forrest}.
This implementation was in turn based on an implementation in Perl by Nikita Borisov\cite{borisov}.
The Perl implementation was heavily inspired by a Modula-2 implementation from 1987 by Luca Cardelli\cite{cardelli}.
Some ideas were also taken from the chapter on types from the programming languages book by Krishnamurthi\cite{plai}.

As mentioned, the type checker is not nearly powerful enough to understand or even represent the entire complexity of Haskell's type system.
However, we have found it to be `good enough' for our purposes.
