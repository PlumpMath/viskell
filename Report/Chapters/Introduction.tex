\chapter{Introduction}

Programming has always been an activity which is done on a computer with a keyboard. Despite the existence of different visual programming languages (e.g. Pure Data) and tools which represent processes visually (e.g. Blender) there are no programs in which well known programming languages like Java, Haskell or C(++) can be visually implemented. 

In this project we have (attempted) to make an application in which the programming language Haskell can be visually implemented. We named our approach ``Viskell''. Viskell tries to aide developers with real-time feedback on Haskell programs. This allows developers to spot type errors directly, without needing to have a compilable program, and provides a visual insight into the program. The intermediate output of a function can also be inspected at runtime. Viskell also allows developers to collaborate and work together on a large multi-touch screen. Other common touch interfaces such as those found on devices like the Microsoft Surface (Pro), are supported as well. Because of the user interface with touch capabilities developers can easily manipulate the code without having a keyboard or mouse around. This provides a better experience for programming on the go. Because of the spatial flexibility offered by handheld devices and touch interfaces, programming with Viskell should be more fun and more ergonomic.

This report discusses different aspects of the development process. The upcoming chapters discuss the requirements analysis, design process, implementation process (front end \& back end), how the application has been tested, the evaluation of the process and the conclusions.
