\chapter{User Guide}
\label{chap:Guide}

Welcome to Viskell!
Viskell is a tool to experiment with the Haskell programming language in a visual and playful way.
You don't need to be a Haskell expert to use Viskell: this User Guide will tell you everything you know to get started.

\textbf{Squares!}
First, we're building a program that calculates the product of two numbers.
To do that, right-click (or tap-and-hold) anywhere to open the \emph{function menu}, if it isn't open already.
(It's the big, purple box: it's hard to miss.)
The function menu has a number of buttons to choose from.
For now, click the button that says \emph{Display Block} to add a display to your program.
The display block doesn't do much on its own: it only displays the value that it's connected to.
Next, add a \emph{value block} by clicking the respective button on the function menu.
You'll be asked to provide a value: something like 1234 will do.
You should now have two blocks: drag the output block so that it's a bit below the value block.
Then, from the \emph{Numeric types} category, click \emph{(*)}, the multiplication function.
The multiplication function has two inputs.
Connect both inputs (the top dots) to the value block you created by clicking and dragging from one dots to the other.
Then attach the function block's output anchor (the bottom dot) to the display block's input.
The square of the number you picked (1522756) should now appear in the output block.
Congratulations!
You've made a real Haskell program!

\textbf{Sliders!}
Our first program isn't very exciting.
You could write the same program in regular Haskell by typing \texttt{5 * 5}.
Viskell has a few tricks up its sleeve, however.
For our second example, we'll try adding a \emph{slider block}.
From the function menu, click the button to get a slider block.
Then connect it to one of the inputs (again, the top dots) of your multiplication function block.
Drag the slider to change its value: you should see the value on the output block change as well.

\textbf{Getting the Haskell code for your program}
Viskell generates Haskell code for you, which you can see by selecting your display block, pressing the \textbf{Z} key, and then selecting the `Haskell Source' tab.
Be careful, though: this is the Haskell code that actually gets executed.
It's probably not `good' Haskell code.
