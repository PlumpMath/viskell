\chapter{User Guide}
\label{chap:Guide}

\section{Developer guide}

This part of the document provides some information to developers on how to use and extend the functionality of the program.

\subsection{Haskell back-end}

\subsubsection{Types}

All types are an instance of a subclass of the \code{Type} class. Haskell types \emph{should} where possible be implemented as a class instance instead of a class itself.
This will mostly be \code{ConstT} as this class can be used for almost anything with a type constructor.

Non-variable types can be easily compared. Two instances of the same type are equal (that is, two instances with the same base class and constructor arguments).
Variable types are only equal if it is the same instance. This is used to avoid confusion between variable types with the same name.

\subsubsection{Expressions}

All kinds of expressions inherit from the \code{Expr} class.

\subsubsection{Catalog}

Extending the catalog should be straightforward.
An XSD (XML Schema Definition) is provided for the catalog XML file describing the layout of the file. Catalog files can be validated against this XSD.

Make sure that the software is capable of handling a type, typeclass or function before (permanently) adding it to the catalog.

\subsection{User interface}
