\chapter{Requirements Analysis}

The first step in creating an application or program is defining the requirements. In our discussions with the client, he emphasized that the main goal of this project was making a user-friendly interface.

\subsection{User}

How to divide the requirements in necessary and desirable requirements depends on the kind of target user for which the application is designed. 
In this project the focus lies mainly on users who already know how to program in Haskell.
The program should aid these users in creating an overview of their program and informing them about (possible) mistakes without the user having to write a complete program and then compile it.

\subsection{Requirements}
For this project, the requirements can be devided in three subgroups, namely: visual requirements, functional requirements and multi-touch requirements. In the following subsections each of the groups will be discussed. In some cases a requirement fits in multiple groups; in these cases the best fitting group has been chosen to contain this requirement.

\subsubsection{Visual requirements} 

As discussed earlier, the user experience is a very important aspect in this project. With this in mind, the visual aspects of the application are important. De visual requirements are the following:

\begin{enumerate}
 \item the structure of the visual program has to logically represent Haskell.
 \begin{enumerate}
 \item[1.1.] Haskell functions are represented by functionblocks.
 \item[1.2.] The program has to be able to show a compact notation in which nesting and stacking of functionblocks is possible.
 \end{enumerate}
 \item
 \item
\end{enumerate}

\subsubsection{Functional requirements}

\subsubsection{Multi-touch requirements}
