\chapter{Test Plan}
\index{test plan}

To test the finished product, we used (automatic) unit tests as well as (manual) acceptance tests.
The unit tests mainly cover the back end of the application because it is difficult to unit test the front end, mainly because of the user interface.
The acceptance tests are focused on the front end.

\section{Acceptance tests} \label{acceptance tests}
	\subsection{Test 1}
		This test evaluates the functioning of the function menu and the spawning of basic blocks.
		
		\begin{enumerate}
			\item Right click on the workspace to open the function menu.
			\item Click on the item labelled \emph{Basic}.
			\item Observe that contents are present in the opened list.
			\item Click on the item again to close the list.
			\item Press the button at the bottom of the menu labelled \emph{Value block}.
			\item Enter \texttt{"Hello"} into the opened dialog window.
			\item Confirm the presence of a block labelled \texttt{"Hello"} in the workspace.
		\end{enumerate}
		
	\subsection{Test 2}
		This test is a follow up to test 1 and will evaluate the functioning of the context menu
		and the delete function.
		
		\begin{enumerate}
			\item Right click on the workspace to open the function menu.
			\item Press the button at the bottom of the menu labelled \emph{Value block}.
			\item Enter \texttt{"DeleteMe"} into the opened dialog window.
			\item Confirm the presence of a block labelled \texttt{"DeleteMe"} in the workspace.
			\item Select the block by left clicking on it.
			\item Right click on the selected block to open the context menu.
			\item Click on the scissor icon.
			\item Confirm that the block has been removed from the workspace.
		\end{enumerate}
		
	\subsection{Test 3}
		This test will evaluate the functioning of the plus operator function block and the display block.
		
		\begin{enumerate}
			\item Right click on the workspace to open the function menu.
			\item Press the button at the bottom of the menu labelled \emph{Value block}.
			\item Enter the number \texttt{5} into the opened dialog window.
			\item Confirm the presence of a block labelled \texttt{5} in the workspace.
			\item Repeat steps 2, 3 and 4.
			\item Click on the item labelled \emph{Numeric types}.
			\item Observe that contents are present in the opened list.
			\item Select the entry in the list labelled \texttt{(+)}.
			\item Confirm that a function block titled \texttt{(+)} has been made in the workspace.
			\item Left click on the bottom anchor on one of the value blocks and drag a line all the way to one of the upper anchors of the function block.
			\item Repeat the previous step for the other value block.
			\item Click on the button at the bottom of the function menu labelled \emph{Display block}.
			\item Confirm that creation of a display block in the workspace.
			\item Left click on the bottom anchor of the function block and drag a line all the way to the upper anchor of the display block.
			\item Confirm that the display block now displays the value \texttt{10}.
		\end{enumerate}
		
	\subsection{Test 4}
		This test will verify that the the type inference of Viskell is working correctly.
		
		\begin{enumerate}
			\item Right click on the workspace to open the function menu.
			\item Press the button at the bottom of the menu labelled \emph{Value block}.
			\item Enter the value \texttt{False} into the opened dialog window.
			\item Click on the item labelled \emph{Numeric types}.
			\item Observe that contents are present in the opened list.
			\item Select the entry in the list labelled \texttt{(+)}.
			\item Confirm that a function block titled \texttt{(+)} has been made in the workspace.
			\item Connect the output of the value block with the function block.
			\item Confirm that the argument in the function block and the connection line are highlighted red.
		\end{enumerate}
	
	\subsection{Test 5}
		This test will evaluate the functionality of higher order functions, knots and graph blocks.
		
		\begin{enumerate}
			\item Right click on the workspace to open the function menu.
			\item Press the button at the bottom of the menu labelled \emph{Graph block}.
			\item Select the entry in the function menu labelled \emph{Numeric types}.
			\item Scroll down in the newly opened list until you find the \texttt{sin} function.
			\item Click on the item to create a new \texttt{sin} function block.
			\item Press the black circular knot and drag it to the left.
			\item If done correctly the function block should now display \texttt{((Floating a) -> (Floating a))}.
			\item Click on the output anchor at the bottom of the function block and drag a line to the input anchor of the graph block.
			\item The graph block should now display a regular sinus.
		\end{enumerate}

	\subsection{Test 6}
		This test will test the workings of function definitions. It will also illustrate the live updating of a graph block.

		\begin{enumerate}
			\item Right click on the workspace to open the function menu.
			\item Select the entry in the function menu labelled \emph{Numeric types}.
			\item Select the \texttt{(*)} function to create a new function block.
			\item Scroll down in the newly opened list until you find the \texttt{sin} function.
			\item Click on the item to create a new \texttt{sin} function block.
			\item Press the button at the bottom of the menu labelled \emph{Graph block}.
			\item Press the button at the bottom of the menu labelled \emph{Slider block}.
			\item Press the button at the bottom of the menu labelled \emph{Definition block}.
			\item In the opened dialog window input: \texttt{trans :: Float -> Float} and press enter.
			\item Left click on the upper input anchor of the definition block and drag a line to the left input anchor of the \texttt{(*)} function.
			\item Left click on the output anchor of the slider block and drag a line to the right input anchor of the \texttt{(*)} function.
			\item Left click on the output anchor of the \texttt{(*)} function and drag a line to the input anchor of the \texttt{sin} function.
			\item Left click on the output anchor of the \texttt{sin} function and connect it to the bottom input anchor of the definition block.
			\item Left click the output anchor of the definition block and connect it to the input anchor of the graph block.
			\item Confirm that the connections between all the blocks are working by observing a red line along the X-axis of the graph block.
			\item Drag the slider of the slider block slowly to the right and observe the output in the graph block.
			\item If done correctly you should have observed the increase in frequency of the \texttt{sin} function.
		\end{enumerate}

\section{Unit tests} \label{unit tests}

A large part of the back end is covered by unit tests.
Almost all tests focus on the Haskell representation to make sure that the Haskell code generated by the program is as expected.
The type parser is also tested extensively to make sure types are consistent and no invalid type errors are raised.

The following parts are covered by the unit tests:

\begin{enumerate}
	\item GHCJ
	\begin{enumerate}
		\item Evaluating expressions
		\item Retrieving expressions
	\end{enumerate}
	\item Haskell catalog
	\begin{enumerate}
		\item Parsing of categories
		\item Parsing of default functions
		\item Parsing of type classes
	\end{enumerate}
	\item Expressions
	\begin{enumerate}
		\item Haskell code representation
		\item Types of expressions
		\item Inferred types of expressions
	\end{enumerate}
	\item Types and type classes
	\begin{enumerate}
		\item Haskell signature representation
		\item Type propagation
		\item Type inference: unification
		\item Type inference: pruning
		\item Deep copying
		\item Equality
	\end{enumerate}
	\item Type parser
	\begin{enumerate}
		\item Parsing of constant types, variable types with and without type classes, function types, list types, tuple types
		\item Parsing of complex, combined types
	\end{enumerate}
\end{enumerate}

The tests are not described in detail here, the exact test cases are included in the source code.
